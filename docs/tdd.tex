\documentclass[12pt,a4paper]{article}
\usepackage[a4paper,margin=1in]{geometry}
\usepackage{graphicx}
\usepackage{hyperref}
\usepackage{titlesec}
\usepackage{enumitem}
\usepackage{longtable}
\usepackage{booktabs}
\usepackage{fancyhdr}

%----------------------------
% Title and Formatting
%----------------------------
\titleformat{\section}{\large\bfseries}{\thesection}{1em}{}
\titleformat{\subsection}{\normalsize\bfseries}{\thesubsection}{1em}{}
\setlist{nosep}

\pagestyle{fancy}
\fancyhf{}
\rhead{Technical Design Document}
\lhead{\leftmark}
\cfoot{\thepage}

\title{\textbf{Technical Design Document}\\[0.5em]\large [Project Title]}

\author{
    [Team Name] \\
    \texttt{[team-email@example.com]}
}

\date{\today}

%----------------------------
\begin{document}
\maketitle
\newpage
\tableofcontents
\newpage

%----------------------------
\section{Overview}
%Describe the purpose of this document and its intended audience.  
%Mention any prerequisite knowledge required before reading.  
%Specify whether it targets a technical or non-technical audience.
The purpose of this document is to present the technical design of [describe your system/project]. \\\\
This document is intended for a technical audience with a background in [relevant fields]. Non-technical readers may find high-level summaries useful but should refer to supplementary documentation at the end of this document for background on the underlying mechanisms employed.
\newpage
%----------------------------
\section{Problem}

[Describe the current state and limitations]

\subsection{The Core Problem}

[Describe the specific problem your system addresses]

\subsection{The Core Goal}

The goal is to develop [describe your solution]:
\begin{itemize}
    \item [Goal 1]
    \item [Goal 2]
    \item [Goal 3]
\end{itemize}
\newpage
%----------------------------
\section{Tenets}
TODO!
List the key principles or beliefs guiding the design decisions.  
Tenets help align teams and establish common ground on critical design questions.
\newpage
%----------------------------
\section{Requirements}
TODO! Describe all requirements imposed by the problem.  
Write from the end-user's perspective. Include user stories or use cases.

\subsection*{Example Use Cases}
\begin{itemize}
  \item As a [user type], I want to [action].
  \item As a [user type], I want [capability].
\end{itemize}
%----------------------------
\subsection{Out of Scope}
TODO! List what is explicitly out of scope to avoid misunderstandings.

%----------------------------
\subsection{Success Criteria}
TODO! Describe how success will be measured once the solution is in production.  
Include measurable metrics such as performance, scalability, or cost efficiency.

\subsection*{Example Metrics}
\begin{itemize}
  \item [Metric 1]
  \item [Metric 2]
  \item [Metric 3]
\end{itemize}
\newpage
%----------------------------
\section{Architecture}
Describe the architecture of the proposed solution using text, diagrams, and bullet points.  
Explain why this design was chosen.

\subsection{High-Level Overview (HLD)}
List all logical system components:
\begin{itemize}
  \item Component 1
  \item Component 2
  \item Component 3
  \item Component 4
  \item Component 5
\end{itemize}

Include diagrams as needed:
\begin{center}
%\includegraphics[width=0.8\textwidth]{architecture-diagram.png}
\end{center}

\subsection{API Design}
List all APIs used for interaction between users or services.  
Include HTTP methods, payloads, versions, and example requests/responses.  
Discuss future evolution of the APIs.

\subsection{Data Storage and Model}
Describe the data model and database choice.  
Estimate data volume, forecast growth, and justify scalability.  
Discuss data pipelines, ingestion, and pre-processing if applicable.

\subsection{Application / Component Level Design (LLD)}
Provide detailed design of each system component, including data flow and control flow diagrams.

\newpage
%----------------------------
\section{Dependencies}
List external and internal dependencies.  
Document assumptions and risks associated with these dependencies.
%----------------------------
\subsection{Design Alternatives Considered}
Discuss alternative designs evaluated.  
Provide a comparison table highlighting trade-offs.

\begin{longtable}{@{}p{3cm}p{5cm}p{5cm}@{}}
\toprule
\textbf{Option} & \textbf{Pros} & \textbf{Cons} \\ \midrule
Option A & [Advantages] & [Disadvantages] \\
Option B & [Advantages] & [Disadvantages] \\ \bottomrule
\end{longtable}

\newpage
%----------------------------
\section{Cost Analysis}
Estimate infrastructure and operational costs.  
Plan for future growth and scalability.

\newpage
%----------------------------
\section{Failures and Risks}
Discuss potential system failures such as:
\begin{itemize}
  \item Dependency failures
  \item Traffic overflow
  \item Performance degradation
  \item Logic bugs
\end{itemize}

Identify risks such as external dependencies or resource limitations.  
Include potential mitigation strategies.

\newpage
%----------------------------
\section{Non-Functional Requirements}
Cover scalability, availability, maintainability, reliability, latency, and security.

\subsection{Scalability}
Expected users, transactions, and data volume.

\subsection{Latency and Availability}
Define SLAs (e.g., P99, P50 latency targets).

\subsection{Maintainability}
Explain maintenance expectations and ownership.

\subsection{Security}
Describe security measures and required compliance levels.

\newpage
%----------------------------
\section{Testing and Observability}
Outline strategies for testing, monitoring, and alerting.

\subsection{Testing}
Explain testing approach: unit, integration, A/B, or stress tests.

\subsection{Metrics and Alarms}
List key performance metrics, dashboards, and alert configurations.

\newpage
%----------------------------
\section{Future Improvements}
List planned features or improvements for future releases.

\newpage
%----------------------------
\section{FAQs}
Include frequently asked questions and their answers.

[Question?]
[Answer here]

\newpage
%----------------------------
\appendix
\section{Appendix A: Subtitle (Optional)}
Include detailed technical analysis or supporting data.

\newpage
%----------------------------
\section{Glossary}
\begin{description}
  \item[Term 1] Definition
  \item[Term 2] Definition
\end{description}

\newpage
%----------------------------
\section{References}
\begin{itemize}
  \item [Reference 1]: \url{https://example.com}
  \item [Reference 2]: \url{https://example.com}
\end{itemize}

\end{document}